\chapter{Specifica dei test}\label{specifica-dei-test}
L'analisi dinamica consiste nell'esecuzione di vari oggetti di prova allo scopo di studiare il comportamento del programma in un insieme finito di casi rappresentativi di tutte le possibili esecuzioni del codice. Ogni esecuzione rappresenta un test, che possono essere di vario tipo, qui di seguito elencati.
Il gruppo SWEasabi ha deciso che, per perseguire la maggior correttezza del prodotto possibile, l'attività di verifica verrà svolta in parallelo allo sviluppo. L'obiettivo da perseguire sarà quindi quello di rendere i test il più automatici possibile allo scopo di non rallentare il processo di sviluppo.

Inoltre, giudicando prematura la definizione dei test in questa fase, il gruppo ha deciso di definirli nelle successive versioni di questo documento.

\section{Test di unità}
I test di unità hanno come scopo quello di dimostrare la correttezza di ciascuna unità individualmente, arrivando al 100\% quando ogni singola unità è stata testata con esito positivo. 

\section{Test di integrazione}
I test di integrazione verificano l'integrazione tra le componenti software, ovvero che due o più unità già testate con esito positivo lavorino insieme in modo corretto generando il risultato e comportamento atteso.

\section{Test di sistema}
I test di sistema si occupano di verificare il comportamento del sistema, controllando che aderisca correttamente ai requisiti software individuati nell'Analisi dei Requisiti.

\section{Test di accettazione}
I test di accettazione costituiscono il collaudo del prodotto, svolto in presenza del committente, con lo scopo di verificare il soddisfacimento di tutti i requisiti richiesti e definiti nell'Analisi dei Requisiti.

\section{Test di regressione}
I test di regressione hanno l'obiettivo di identificare eventuali errori ausati da modifiche introdotte in nuove versioni del prodotto, verificando quindi che parti del sistema già testate non vengano danneggiate da nuovi componenti.

\section{Stato dei test}

\begin{center}
    \begin{xltabular}{\linewidth}{|l|X|l|}
        \hline
        \textbf{Tipologia} & \textbf{Descrizione} & \textbf{Stato}\\
        \hline
        Unità & Si verifica il corretto funzionamento del metodo di autenticazione in caso di credenziali corrette fornite & Superato\\
        Unità & Si verifica il corretto funzionamento del metodo di autenticazione in caso di credenziali errate fornite & Non superato\\
        Unità & Si verifica il corretto funzionamento dei verificatori dei token & Superato\\
        Unità & Si verifica che il metodo \texttt{issueRefreshToken} produca il risultato atteso & Superato\\
        Unità & Si verifica che il metodo \texttt{issueAccessToken} produca il risultato atteso & Superato\\
        Unità & Si verifica che il metodo \texttt{updateRefreshToken} produca il risultato atteso & Superato\\
        Unità & Si verifica il corretto funzionamento del metodo per invalidare un token & Non superato\\
        Unità & Si verifica che la creazione della classe JwtExtractor avvenga in modo corretto in caso di campi non validi & Superato\\
        Unità & Si verifica che l'estrazione dello username produca il risultato atteso & Superato\\
        Unità & Si verifica che la creazione della classe JwtPackager avvenga in modo corretto in caso di campi non validi & Non superato\\
        Unità & Si verifica che il metodo \texttt{pack} produca il risultato atteso & Superato\\
        Unità & Si verifica che la creazione della classe LoginResult avvenga in modo corretto in caso di campi non validi & Non superato\\
        Unità & Si verifica il corretto funzionamento dei metodi che ritornano le informazioni richieste & Superato\\
        Unità & Si verifica il corretto funzionamento del metodo per il controllo di un utente in blacklist & Superato\\
        Unità & Si verifica il corretto funzionamento del metodo per aggiungere un utente alla blacklist & Non superato\\
        Unità & Si verifica il corretto funzionamento dei metodi per il recupero delle chiavi & Superato\\
        Unità & Si verifica il corretto funzionamento dei metodi per il recupero della password (hash) & Superato\\
        Unità & Si verifica il corretto funzionamento del metodo per eseguire il logout & Superato\\

        % Unità &  & Non implementato\\
        \hline
        Integrazione & Descrizione Test & Non implementato\\
        Integrazione & Descrizione Test & Non implementato\\
        Integrazione & Descrizione Test & Non implementato\\
        Integrazione & Descrizione Test & Non implementato\\
        \hline
        Sistema & Il sistema deve mostrare un messaggio d'errore esplicativo all'utente in caso di errore di autenticazione & Non implementato\\
        Sistema & Ogni cambiamento di stato di un lampione deve apparire automaticamente nell'interfaccia utente & Non implementato\\
        Sistema & Deve essere possibile aggiungere nuovi sensori di luminosità o presenza a sistema & Non implementato\\
        Sistema & L'intensità luminosa di un'area di illuminazione deve poter essere gestita manualmente da un utente & Non implementato\\
        Sistema & Il sistema deve accendere un'area per un lasso di tempo pre-configurato quando rileva persone in prossimità della stessa & Non implementato\\
        Sistema & Il sistema deve riportare l'intensità luminosa dell'area al valore di default una volta passato il tempo impostato & Non implementato\\
        Sistema & L'utente deve effettuare l'accesso per poter utilizzare le funzionalità del sistema & Non implementato\\
        Sistema & L'utente deve poter inserire la locazione geografica del sensore nel sistema & Non implementato\\
        Sistema & L'utente deve poter inserire il raggio d'azione del sensore nel sistema & Non implementato\\
        Sistema & L'utente deve essere in grado di visualizzare quali aree sono illuminate in un dato momento & Non implementato\\
        Sistema & L'utente deve poter impostare la modalità dell'area di illuminazione come manuale & Non implementato\\
        Sistema & L'utente deve poter impostare la modalità dell'area di illuminazione come automatica & Non implementato\\
        Sistema & L'utente deve poter modificare la luminosità dei lampioni del sistema a livello globale & Non implementato\\
        Sistema & Il sistema deve essere in grado di regolare automaticamente l'intensità luminosa dei lampioni in caso rilevi valori fuori soglia di luminosità ambientale & Non implementato\\
        Sistema & Il sistema deve essere in grado di rilevare automaticamente guasti nei lampioni & Non implementato\\
        Sistema & L'utente deve essere in grado di segnalare manualmente guasti nei lampioni & Non implementato\\
        Sistema & Il gestore deve poter visualizzare l'elenco degli impianti guasti & Non implementato\\
        Sistema & L'utente manutentore deve essere in grado di chiudere le segnalazioni di guasto dalla lista guasti & Non implementato\\
        Sistema & Il sistema deve essere in grado di ricevere informazioni dal sensore in modalità Push & Non implementato\\
        Sistema & Il sistema deve essere in grado di ricevere informazioni dal sensore in modalità Pull & Non implementato\\
        Sistema & L'applicazione deve essere visualizzabile su dispositivi mobile & Non implementato\\
        Sistema & L'applicazione client deve poter essere utilizzata sulla versione 110.0 di Chrome  & Non implementato\\
        Sistema & L'applicazione client deve poter essere utilizzata sulla versione 110.0 di Firefox  & Non implementato\\
        Sistema & L'applicazione client deve poter essere utilizzata sulla versione 16.3 di Safari  & Non implementato\\
        Sistema & L'applicazione client deve essere conforme almeno al livello AA delle WCAG  & Non implementato\\
    
        
        \hline
    \end{xltabular}
\end{center}