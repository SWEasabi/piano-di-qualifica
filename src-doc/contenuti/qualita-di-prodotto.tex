\chapter{Qualità di prodotto}\label{qualita-di-prodotto}

\section{Introduzione}
Utilizzando lo standard ISO/IEC 12207:1995 , abbiamo individuato le Qualità che si ritengono necessarie durante l'intero ciclo di vita del prodotto, definendo metriche e obiettivi specifici al loro conseguimento.

Sono questi riportati di seguito.

\section{Affidabilità}
\begin{itemize}
    \item \textbf{Descrizione}
    \begin{itemize}
        \item L'affidabilità del prodotto si riferisce alla sua capacità di svolgere le sue funzioni, a prescidere dal manifestarsi di errori, provando ad eliminarne la loro incidenza.
    \end{itemize}
    
    \item \textbf{metriche}
    \begin{itemize}
        \item MPD01
        \begin{itemize}
            \item \textbf{Spiegazione}: Misurazione dei difetti del prodotto;
        \item \textbf{Metrica di misurazione}: Percentuale;
        \item \textbf{Valore ottimale}: 80\%;
        \item \textbf{Soglia accettabile}: 60\%;
        \item \textbf{Note}: I valori potranno essere modificati .
        \end{itemize}
    \end{itemize}
    \end{itemize}

\section{Efficienza}
\begin{itemize}
    \item \textbf{Descrizione}
    \begin{itemize}
        \item L'efficienza del prodotto si riferisce alla sua capacità di svolgere le sue funzioni utilizzando per completare queste il minor numero di risorse.
    \end{itemize}
    
    \item \textbf{metriche}
    \begin{itemize}
        \item MPD02
        \begin{itemize}
\item \textbf{Spiegazione}: Tempo medio di risposta;
        \item \textbf{Metrica di misurazione}: Secondi;
        \item \textbf{Valore ottimale}: 5 Secondi;
        \item \textbf{Soglia accettabile}: 7 Secondi.
        \end{itemize}
    \end{itemize}
    \end{itemize}

\section{Funzionalità}
\begin{itemize}
    \item \textbf{Descrizione}
    \begin{itemize}
        \item La Funzionalità del prodotto si riferisce alla capacità del prodotto di svolgere le funzioni previste in modo completo e corretto.
    \end{itemize}
    
    \item \textbf{metriche}
    \begin{itemize}
        \item MPD03
        \begin{itemize}
            \item \textbf{Spiegazione}: Copertura dei requisiti;
            \item \textbf{Metrica di misurazione}: Percentuale;
            \item \textbf{Valore ottimale}: 100\% dei requisiti obbligatori e 80\% dei requisiti opzionali;
            \item \textbf{Soglia accettabile}: 100\% dei requisiti obbligatori.
        \end{itemize}
    \end{itemize}
    \end{itemize}
    
\section{Manutenibilità}
\begin{itemize}
    \item \textbf{Descrizione}
    \begin{itemize}
        \item La Manutenibilità del prodotto si riferisce alla capacità del prodotto di essere modificato e mantenuto in modo efficiente.
    \end{itemize}
    
    \item \textbf{metriche}
    \begin{itemize}
        \item MPD04
        \begin{itemize}
            \item \textbf{Spiegazione}: Comprensibilità del codice;
        \item \textbf{Metrica di misurazione}: Percentuale;
        \item \textbf{Valore ottimale}: 85\% - 100\% ;
        \item \textbf{Soglia accettabile}: 65\% .
        \end{itemize}
    \end{itemize}
    \end{itemize}


\section{Portabilità}
\begin{itemize}
    \item \textbf{Descrizione}
    \begin{itemize}
        \item La Portabilità del prodotto si riferisce alla capacità del prodotto di essere utilizzato in diverse piattaforme e ambienti.
    \end{itemize}
    
    \item \textbf{metriche}
    \begin{itemize}
        \item MPD05
        \begin{itemize}
            \item \textbf{Spiegazione}: Compatibilità del prodotto;
        \item \textbf{Metrica di misurazione}: Percentuale;
        \item \textbf{Valore ottimale}: 85\% - 100\%;
        \item \textbf{Soglia accettabile}: 60\%.
        \end{itemize}
    \end{itemize}
    \end{itemize}



\section{Usabilità}

    \begin{itemize}
    \item \textbf{Descrizione}
    \begin{itemize}
    \item L'Usabilità del prodotto si riferisce alla capacità del prodotto di essere utilizzato in modo efficace, efficiente e soddisfacente dagli utenti finali.
    \end{itemize}

    \item MPD06
        \begin{itemize}
            \item \textbf{Spiegazione}: Facilità d'uso del prodotto;
            \item \textbf{Metrica di misurazione}: Numero di errori compiuti dagli utenti durante l'utilizzo del prodotto;
            \item \textbf{Valore ottimale}: Inferiore a 1 errore per utente;
            \item \textbf{Soglia accettabile}: Inferiore a 2 errori per utente.
        \end{itemize}
    \end{itemize}
    
