\chapter{Introduzione}

\section{Glossario}
Un glossario utile con alcune definizioni per lavorare al progetto al fine di chiarire eventuali termini che possono generare dubbi interpretativi:

\section{Scopo del documento}
Lo scopo di questo documento è quello di presentare in modo ordinato e scorrevole gli standard di qualità del gruppo, basandosi su caratteristiche misurabili tramite metriche oggettive. Questi standard verranno implementati con l'obiettivo di conseguire un miglioramento continuo attraverso azioni correttive, e non semplicemente come misurazioni statiche. All'interno del documento verranno inoltre dettagliati test e attività di verifica utilizzate per rilevare la qualità descritta.

Le metriche di qualità si possono individuare tramite le seguenti sigle: \textbf{M - Sigla - Numero}

dove le diverse parti indicano:

\begin{itemize}
    \item \textbf{M} indica che si tratta di una metrica;
    \item \textbf{Sigla} si riferisce alla tipologia di metrica, in particolare:
    \begin{itemize}
        \item \textbf{PD} indica prodotto;
        \item \textbf{PC} indica processo;
        \item \textbf{T} indica test.
    \end{itemize}
    \item \textbf{Numero} è il numero identificativo della metrica, a partire da 01.
\end{itemize}

\section{Fonti}
Le diverse fonti utilizzate includono:
\begin{itemize}
    \item \textbf{Slide T02} del corso di Ingegneria del Software, consultabile al link: \href{https://www.math.unipd.it/~tullio/IS-1/2022/Dispense/T02.pdf}{Slide T02};
    \item \textbf{Slide T12} del corso di Ingegneria del Software, consultabile al link: \href{https://www.math.unipd.it/~tullio/IS-1/2022/Dispense/T12.pdf}{Slide T12};
    \item \textbf{Pagina Metriche di Progetto} di Wikipedia, consultabile al link: \href{https://it.wikipedia.org/wiki/Metriche_di_progetto}{Metriche di Progetto};
    \item \textbf{Pagina Software Metrics} di Wikipedia (inglese), consultabile al link: \href{https://en.wikipedia.org/wiki/Software_metric}{Software Metric};
    \item \textbf{Pagina Indice di Gulpease} di Wikipedia, consultabile al link: \href{https://it.wikipedia.org/wiki/Indice_Gulpease}{Indice Gulpease}.
\end{itemize}