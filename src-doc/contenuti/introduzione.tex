\chapter{Introduzione}
\section{Scopo del documento}
Lo scopo di questo documento è quello di presentare in modo ordinato e scorrevole gli standard di qualità \textit{SWEasabi team}, basandosi su caratteristiche misurabili tramite metriche oggettive. Questi standard verranno implementati con l'obiettivo di conseguire un miglioramento continuo attraverso azioni correttive, e non semplicemente come misurazioni statiche. All'interno del documento verranno inoltre dettagliati test e attività di verifica utilizzate per rilevare la qualità descritta.

Le metriche di qualità si possono individuare tramite le seguenti sigle: \textbf{M - Sigla - Numero}

dove le diverse parti indicano:

\begin{itemize}
    \item \textbf{M} indica che si tratta di una metrica;
    \item \textbf{Sigla} si riferisce alla tipologia di metrica, in particolare:
    \begin{itemize}
        \item \textbf{PD} indica prodotto;
        \item \textbf{PC} indica processo;
        \item \textbf{T} indica test.
    \end{itemize}
    \item \textbf{Numero} è il numero identificativo della metrica, a partire da 01.
\end{itemize}

\section{Scopo del prodotto}
Il risparmio delle risorse del Pianeta e in particolare delle fonti energetiche è entrato con forza nell'agenda politica dell'Unione Europea. Fra gli impatti più evidenti, spicca la crescita esponenziale del prezzo del gas, risorsa ancora largamente utilizzata come materia prima per la produzione di energia elettrica.
Per far fronte al rincaro delle bollette energetiche, molti comuni italiani stanno annunciando tagli all'illuminazione pubblica, che necessita di una quantità considerevole di energia elettrica.

Il capitolato C2, \textit{Lumos Minima}, pone come obiettivo lo sviluppo di un sistema per l'ottimizzazione dell'illuminazione pubblica che permetta ai gestori di sfruttare la possibilità di regolare l'intensità della luce emessa dagli impianti di illuminazione.

\section{Glossario}
Un glossario utile con alcune definizioni per lavorare al progetto al fine di chiarire eventuali termini che possono generare dubbi interpretativi:

\section{Maturità del documento}
Il presente documento è redatto con un approccio incrementale in modo tale da trattare modifiche o aggiunte in modo efficiente. Non può pertanto essere considerato definitivo nella sua attuale versione.

\section{Riferimenti e richiami}
\subsection{Riferimenti normativi}
\begin{itemize}
    \item Capitolato d'appalto C2 - Lumos Minima: \\ \url{https://www.math.unipd.it/~tullio/IS-1/2022/Progetto/C2.pdf}
\end{itemize}

\subsection{Riferimenti informativi}
\begin{itemize}
    \item Slide T02 del corso di Ingegneria del Software: \\ \url{https://www.math.unipd.it/~tullio/IS-1/2022/Dispense/T02.pdf}
    \item Slide T12 del corso di Ingegneria del Software: \\ \url{https://www.math.unipd.it/~tullio/IS-1/2022/Dispense/T12.pdf}
    \item Metriche di Progetto: \\ \url{https://it.wikipedia.org/wiki/Metriche_di_progetto}
    \item Software Metrics: \\ \url{https://en.wikipedia.org/wiki/Software_metric}
    \item Indice di Gulpease: \\ \url{https://it.wikipedia.org/wiki/Indice_Gulpease}
\end{itemize}